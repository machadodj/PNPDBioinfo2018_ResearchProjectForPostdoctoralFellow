\begin{titlepage}
    \author{
        \parbox{\linewidth}{\flushleft \fontsize{11}{14} \sffamily \bfseries 
%            Supervisor: Prof. Dr. Taran Grant\\
            Candidato: Dr. Denis Jacob Machado\\
            Instituição Sede: Instituto de Biociências, Universidade de São Paulo%
        } % end parbox
    } % end author
    \title{
        {\fontsize{14}{17} \sc \sffamily
            Interunidades em Bioinformática % Intituição
        }\\[-2mm]
        {\fontsize{14}{17} \sffamily
            Projeto de Pós-Doutorado (PNPD) % Tipo
        }\\[10mm]
        {\color{RoyalBlue} \sffamily
            Transcriptômica comparada e a defesa química associada à tetrodotoxina no tritão \textit{Taricha granulosa} % Título
        }
        \date{} % empty date
    } % end title
    \settitle \thispagestyle{empty}
    \renewcommand{\abstractname}{\color{RoyalBlue} \fontsize{12}{15} \bfseries \sffamily {Resumo}} % Resumo em português
    \begin{abstract}
    
A tetrodotoxina (TTX) é uma neurotoxina de origem marinha que toma seu nome de peixes tóxicos da Família Tetraodontidae, como o baiacu. Desde sua descoberta, cientistas reportaram TTX em tritões do gênero \textit{Taricha} nos anos de 1960 e em diversos outros filos filogeneticamente distintos, incluingo outros anfíbios,  moluscos marinhos, planárias terrestres, entre outros. Entretanto, mesmo que a pesquisa em TTX seja um campo prolífico alimentado em grande parte pelo seu potencial como inibidor canais de sódio voltagem-dependentes e como agente terapêutico para tratamento de dor, sua biossíntese permanece desconhecida. Esta proposta tem como objetivo ser a primeira em aplicar métodos em bioinformática sobre este problema. Introduzimos aqui resultados preliminares de estudo de transcriptômica comparada de diferentes populações de \textit{Taricha} portadoras da toxina que mostram que estamos próximos de poder apresentar uma lista de canditados de genes potenciamente ligados à defesa química por TTX. Apresentamos também um plano de pesquisa que almezja usar dados transcriptômicos (já disponíveis em parte) de outros anfíbios e de baiacus que tenham TTX para finalmente começarmos a desvendar as bases da defesa química baseada nesta neurotoxina.

    \end{abstract}
    %
    \clearpage
\end{titlepage}