\begin{titlepage}
    \author{
        \parbox{\linewidth}{\flushleft \fontsize{11}{14} \sffamily \bfseries 
            % Supervisor: Professor Taran Grant, Ph.D.\\
            Applicant: Denis Jacob Machado, Ph.D.\\
            Host Institution: Instituto de Biociências, Universidade de São Paulo%
        } % end parbox
    } % end author
    %
    \title{
        {\fontsize{14}{17} \sc \sffamily
            Interunidades em Bioinformática % Intituição
        }\\[-2mm]
        {\fontsize{14}{17} \sffamily
            Projeto de Pós-Doutorado (PNPD) % Tipo
        }\\[10mm]
        {\color{RoyalBlue} \sffamily
            Comparative transcriptomics and tetrodotoxin chemical defense in the rough-skinned newt, \textit{Taricha granulosa} % Título
        }
        \date{} % empty date
    } % end title
    %
    \maketitle \thispagestyle{empty} % \setcounter{page}{+2}
    %
    \renewcommand{\abstractname}{\color{RoyalBlue} \fontsize{12}{15} \bfseries \sffamily {Abstract}} % Resumo em inglês
    \begin{abstract}
    
% Tetrodotoxin (TTX) is a naturally occurring neurotoxin that takes its name from puffers of the Family Tetraodontidae. Since its discovery, scientists reported TTX in amphibians from salamandrid salamanders of the genus Taricha in the 1960s and its now known to occur in many other phyla with no phylogenetic propinquity, from marine mollusks to terrestrial flatworms. However, despite prolific research on TTX that is powered in large part by its relevance as an inhibitor of voltage-activated sodium channels and as a therapeutic agent for pain, its biosynthesis remains unknown. The current research proposal aims to be the first to approach this problem using bioinformatic methods. Here we introduce preliminary data on the comparative transcriptome analysis of different populations of TTX-possessing salamanders that indicates that we are close to proposing a candidate list of potential genes associated with TTX chemical defense in salamanders. Furthermore, we present a research plan to leverage from transcriptome data (some already available) from TTX-possessing pufferfish and frogs that will allow us to finally start unveiling the genetic bases of chemical defense based on TTX.

Tetrodotoxin (TTX) is a naturally occurring neurotoxin that takes its name from puffers of the Family Tetraodontidae. Since its discovery, scientists reported TTX in diverse eukaryotes including dinoflagellates, flatworms, sea slugs, crabs, a starfish, and an octopus. However, amphibians are the only other vertebrate besides pufferfish in which scientists confirmed the presence of TTX. The first report of TTX in amphibians found the toxin in salamandrid salamanders of the genus \textit{Taricha} in the 1960s. We now know that TTX occurs in 27 species of amphibians and is suspected (by phylogeny and aposematic coloration and behavior) to occur in many more. However, despite prolific research on TTX that is powered in large part by its relevance as an inhibitor of voltage-activated sodium channels and as a therapeutic agent for pain, its biosynthesis remains unknown. The current research proposal aims to be the first to approach this problem using bioinformatic methods. Here we introduce preliminary data on the comparative transcriptome analysis of different populations of TTX-possessing salamanders that indicates that we are close to proposing a candidate list of potential genes associated with TTX chemical defense in these animals. Furthermore, we present a research plan to leverage from transcriptome data (some already available) from TTX-possessing pufferfish and frogs that will allow us to finally start unveiling the genetic bases of TTX chemical defense accross different eukaryote taxa.
    
    \end{abstract}
    \clearpage
\end{titlepage}

% \setcounter{page}{+3}
\setcounter{page}{1}