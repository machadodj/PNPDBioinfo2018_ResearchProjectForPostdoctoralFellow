\section{Scientific and Techical Challenges}


This project will require handling biological material from multiple countries. Also, we will execute different protocols for handling RNA samples and quantifying TTX in different locations. Dr. Brodie will handle biological material from newts at the Utah State University, but it is likely that RNA extraction will require a personal visit to his laboratory. Dr. Grant will handle biological material from frogs, including licenses to collect samples and permits to ship them to the sequencing facility. We will send skin samples from amphibians to Dr. Saporito that will quantify the amount of TTX at the John Carroll University.

This project includes both low- and high-risk components. Merely sequencing and annotating as much of the skin transcriptomes of TTX-possessing amphibians will return a significant payoff in terms of new knowledge and possibly descriptive articles, and it is relatively low-risk insofar as it builds on data and information we already possess and requires techniques we are already familiar with (e.g., histology, transmission electron microscopy). In contrast, the more technically challenging aspect of the project will be to compare different transcriptomes in the absence of references and try to identify differentially expressed genes that are common among different TTX-possessing taxa. Our stable relationship with Dr.Janies and his team of bioinformaticians at the University of North Carolina at Charlotte will be constructive in discussing possible solutions to the potential obstacles that may arise, with the possibility of creating new dedicated programs and pipelines that could potently assist the research community interested in the processing transcriptomic data from non-model organisms.