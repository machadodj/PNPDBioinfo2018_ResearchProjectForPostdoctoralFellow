\section{Dissemination and Evaluation}

Dissemination of results will be achieved primarily through scientific papers. Conferences and meetings in the areas of bioinformatics, herpetology, and toxicology will also be appropriate forums to present and discuss the on-going work. Since this is a multidisciplinary project, there are several research fields in which the progress of the work can be subject to discussion with peers. To increase the visibility of this research project, I will showcase my findings at the Laboratório de Anfíbios website (http://www.ib.usp.br/grant/anfibios), which receives an average of 15 unique visits per day from 34 countries. Furthermore, I have experience disseminating my research in the form of graduation courses, including the first course on bioinformatics of the Universidad del Magdalena (40h in 2017) and Graduate Program in Zoology of the Institute of Biosciences of the University of São Paulo (60h in 2018). I have been invited by the Graduate Program in Zoology to give other courses on bioinformatics, and I intend to include examples from my postdoctoral research into the course material as incorporating examples from personal investigations has proven to be an effective way to publicize my work and engage the academic audience.