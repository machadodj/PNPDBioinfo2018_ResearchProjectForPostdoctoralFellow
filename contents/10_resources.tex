\section{Additional Resources}

This project is the next step in the research we initiated during my Ph.D. studies at the Interunits Graduate Program in Bioinformatics in collaboration with the Laboratório de Anfíbios (University of São Paulo). On the basis of the potential of this research area and the need of a multidisciplinary approach with experts in bioinformatics and alkaloid chemical defense, we formed a collaborative network composed of my former dissertation advisor Taran Grant and our colleague in the USA including \href{http://www.biology.usu.edu/about/faculty/edmund-brodie}{Edmund Brodie Jr.} (Utah State University, email: brodie@biology.usu.edu), \href{http://sites.jcu.edu/biology/professor/ralph-a-saporito/}{Ralph A. Saporito} (John Carroll University, email: rsaporito@jcu.edu), and \href{https://cci.uncc.edu/directory/janies-daniel}{Daniel Janies} (Univesity of North Carolina at Charlotte, email: djanies@uncc.edu).

Brodie is a world authority on the convergent evolution of TTX, at both the phenotypic and genetic levels, and on the coevolutionary arms races between amphibians and their snake predators. Saporito is an expert on Chemical Ecology, Tropical Ecology, Behavior, and Evolution, with an emphasis on predator-prey interactions and aposematism. Janies is a national principal investigator in the Tree of Life program (\href{http://echinotol.org/}{http://echinotol.org/}) of the National Science Foundation and the Defense Applied Research Projects Agency funds part of his research. His work involves empirical studies of organismal diversity and development of software, such as Supramap (\href{https://supramap.herokuapp.com/}{https://supramap.herokuapp.com/}).

During my Ph.D., the collaborations above provided biological material and funding for RNA sequencing experiments, travel expenses, field trips, and minor pieces of equipment and reagents for laboratory work. The approval of the current project will help keep these collaborations fruitful for involved. Furthermore, our partnership with Janies and the Bioinformatics and Genomics Department at UNC Charlotte guarantees additional computational resources and access to their sequencing facilities and specialized personnel.

In combination with the resources available at the Laboratório de Anfíbios and the funding that Grant receives from FAPESP, these resources guarantee the feasibility of the current proposal.